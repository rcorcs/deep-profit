\newcommand{\hack}[1]{#1}
\newenvironment{senumerate}{\begin{enumerate}[noitemsep,nolistsep]}{\end{enumerate}}
\newenvironment{sitemize}{\begin{itemize}[noitemsep,nolistsep]}{\end{itemize}}

%\linespread{0.97}


%% \documentclass[10pt,nocopyrightspace]{sigplanconf}


% The following \documentclass options may be useful:

% preprint      Remove this option only once the paper is in final form.
% 10pt          To set in 10-point type instead of 9-point.
% 11pt          To set in 11-point type instead of 9-point.
% authoryear    To obtain author/year citation style instead of numeric.
\usepackage{fancyhdr}
\usepackage{amsmath}

%% BEGIN PREAMPLE
\usepackage{graphicx}
\usepackage{dblfloatfix}
\usepackage{amsmath}
\usepackage{algorithm}
\usepackage{algorithmic}
\usepackage{amssymb}
\usepackage{float}
%\usepackage{subfigure}
\usepackage{subcaption}
\usepackage{multirow}           %tables
\usepackage{rotating}           %tables
\usepackage{enumitem}
\usepackage{calc}
%\usepackage{balance}
\usepackage{url}
%% \usepackage[]{hyperref} %FOR CAMERA READY NO BOOKMARKS!
%\usepackage[bookmarks=false,draft]{hyperref}
\usepackage{xspace}

%\usepackage{minted}
%\usemintedstyle{friendly}

\floatstyle{plain}              %USE \begin{myfloat}...stuff...\end{myfloat} to put stuff into a float
\newfloat{myfloat}{thp}

%% Listing related stuff
\usepackage{listings}
\usepackage{courier}            %Required for pretty listings (more dense)

%% EMACS colors for listing:
\usepackage{color}
%% \definecolor{sh_comment}{rgb}{0.65, 0.00, 0.00 } % red-ish
%% \definecolor{sh_keyword}{rgb}{0.37, 0.69, 0.69}  % blue-green
%% \definecolor{sh_string}{rgb}{0.08, 0.69, 0.08} % green

\definecolor{sh_comment}{rgb}{0.00, 0.50, 0.00 } % green
\definecolor{sh_keyword}{rgb}{0.00, 0.00, 0.60 }  % blue-green
\definecolor{sh_string}{rgb}{0.40, 0.20, 0.40 } % green


%% This manipulates the listing numbering. It decreases it by one at the point used.
\newcommand*\lstDNumber{\addtocounter{lstnumber}{-1}}

\setlength{\dbltextfloatsep}{8pt}
\setlength{\textfloatsep}{8pt}

%% Change from Listing->Algorithm
%% \renewcommand\lstlistingname{Algorithm} % Fix listing caption name: Listing -> Algorithm
%% \renewcommand\lstlistlistingname{Algorithms}
%% \def\lstlistingautorefname{Alg.}

%% \renewcommand{\thelstlisting}{\thesection-\arabic{lstlisting}}
%\renewcommand{\thelstlisting}{\arabic{lstlisting}} % Fix listing numbering Algorithm 1.1 -> Algorithm 1.

%% \renewcommand{\labelenumi}{\roman{enumi}.}

\lstset{
float=[*],
language=C,                % choose the language of the code
%% basicstyle=\scriptsize\ttfamily,
 basicstyle=\scriptsize\ttfamily,
 stringstyle=\color{sh_string},
 keywordstyle = \color{sh_keyword}\bfseries,
 commentstyle=\color{sh_comment}\itshape,
numbers=left,                   % where to put the line-numbers
%% numberstyle=\scriptsize,        % the size of the fonts that are used for the line-numbers
numberstyle=\scriptsize,        % the size of the fonts that are used for the line-numbers
stepnumber=1,                   % the step between two line-numbers. If it is 1 each line will be numbered
numbersep=5pt,                  % how far the line-numbers are from the code
backgroundcolor=\color{white},  % choose the background color. You must add \usepackage{color}
showspaces=false,               % show spaces adding particular underscores
showstringspaces=false,         % underline spaces within strings
showtabs=false,                 % show tabs within strings adding particular underscores
xleftmargin=2em,                % Left margin
frame=lines,                   % adds a frame around the code. none|leftline|topline|bottomline|lines|single|shadowbox
framexleftmargin=1.5em,         % Margin from frame to line numbers
framexbottommargin=0em,         % Distance from last text to frame
morekeywords={in,not,and,or},
%% prebreak=\mbox{\tiny$\searrow$},
prebreak=\space,                % Line break: insert this character at the end of the top line
%% postbreak=\mbox{{\color{blue}\scriptsize$\hookrightarrow$}}, % Line break: blue arrow at the beginning of the bottom line
postbreak=\mbox{{\color{blue}\scriptsize$\hookrightarrow$}}, % Line break: blue arrow at the beginning of the bottom line
breaklines=true,                % sets automatic line breaking
breakatwhitespace=false,        % sets if automatic breaks should only happen at whitespace
tabsize=2,                      % sets default tabsize to 2 spaces
captionpos=t,                   % sets the caption-position to top
escapeinside={@}{@}             % Escape sequence @ESCAPED STUFF@
}

%% Prevent footnotes from spanning multiple pages
\interfootnotelinepenalty=10000

%% Tables
\usepackage{colortbl}
\usepackage{color}
\usepackage{booktabs}
%\usepackage[table]{xcolor}
\definecolor{oddcolor}{gray}{1.0}
\definecolor{evencolor}{gray}{0.9}

%% \newcommand{\refsec}[1]{section~\ref{#1}}
\newcommand{\refSec}[1]{Section~\ref{#1}}
\newcommand{\refSecs}[2]{Sections~\ref{#1}~and~\ref{#2}}
%% \newcommand{\reffig}[1]{figure~\ref{#1}}
\newcommand{\refFig}[1]{Figure~\ref{#1}}
%% \newcommand{\reffigs}[2]{figures~\ref{#1} and~\ref{#2}}
\newcommand{\refFigs}[2]{Figures~\ref{#1} and~\ref{#2}}
\newcommand{\refFigss}[3]{Figures~\ref{#1},~\ref{#2} and~\ref{#3}}
%% \newcommand{\reftab}[1]{table~\ref{#1}}
\newcommand{\refTab}[1]{Table~\ref{#1}}
%% \newcommand{\refalg}[1]{algorithm~\ref{#1}}
\newcommand{\refAlg}[1]{Algorithm~\ref{#1}}

\newcommand{\reflst}[1]{listing~\ref{#1}}
\newcommand{\refLst}[1]{Listing~\ref{#1}}

\newcommand{\CostDiff}{$\mathit{CostDiff}$}
\newcommand{\TotalCost}{$\mathit{TotalCost}$}
\newcommand{\VectorCost}{$\mathit{VectorCost}$}
\newcommand{\ScalarCost}{$\mathit{ScalarCost}$}

\newcommand{\todo}[1]{{\color{red}\large{#1}}}
%\newcommand{\fixme}[1]{{\color{blue}\fbox{FIXME:} #1}}
\newcommand{\update}{{\color{magenta}\fbox{Needs Update.}}}
\newcommand{\code}[1]{{\ttfamily #1}}

\newcommand{\fixme}[1]{{\color{red}{#1}}}

%% END OF PREAMPLE
